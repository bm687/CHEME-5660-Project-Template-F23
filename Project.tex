\documentclass[10pt,twocolumn,twoside,final]{IEEEtran}

% cite package, to clean up citations in the main text. Do not remove.
\usepackage{cite}
\usepackage{lastpage,fancyhdr,graphicx}
\usepackage{amssymb,amsmath}
\usepackage{mathtools}
\usepackage{hyperref}
\usepackage[square,sort,comma,numbers]{natbib}
\renewcommand{\bibfont}{\footnotesize}


% Remove brackets from numbering in List of References
\makeatletter
\renewcommand{\@biblabel}[1]{\quad#1.}
\makeatother

% *** SUBFIGURE PACKAGES ***
\ifCLASSOPTIONcompsoc
  \usepackage[caption=false,font=footnotesize,labelfont=sf,textfont=sf]{subfig}
\else
  \usepackage[caption=false,font=footnotesize]{subfig}
\fi

\begin{document}

\title{Strategic Financial Planning for Sustainable Pediatric Healthcare: A Comprehensive Analysis and Endowment Structure in Kampala, Uganda}


% author names and affiliations
% use a multiple column layout for up to three different
% affiliations
\author{\IEEEauthorblockN{Baisampayan Moitra\IEEEauthorrefmark{1},
\\
\IEEEauthorblockA{\IEEEauthorrefmark{1}Meinig School of Biomedical Engineering - Cornell University, Ithaca, NY 14850 USA }

\maketitle

\begin{abstract}
  This report outlines the financial and investment strategy devised by our non-governmental organization (NGO)- Solo Jacket in response to a generous grant from the Bill and Melinda Gates Foundation. The grant is intended to establish and sustain pediatric health services in the greater Kampala, Uganda, area for a decade. Our project involves constructing a central facility in Kampala and 15 satellite clinics within a 100-mile radius, aiming to ensure the provision of quality healthcare to the community. The objectives of the project include identifying the capital and operating costs for the ten-year term, structuring an endowment to cover annual expenses and 20 % reinvestment payment, and quantifying the initial grant size while developing an investment portfolio to meet financial obligations.
  To achieve these objectives, the report first outlines the anticipated capital and operating costs, taking into account the construction and maintenance of facilities, staffing expenses, and other operational overheads. A continuous discounting approach with a 5% discount rate is applied to assess the present value of these costs over the ten-year period, starting from the project's initiation on January 3, 2023.
  Subsequently, the report proposes a financial strategy involving the creation of an endowment to cover annual operating expenses and the 20% reinvestment payment. The endowment is structured to ensure sustainability and repay the initial face value to the Gates Foundation at the end of the ten-year term
\end{abstract}


\begin{IEEEkeywords}
  NGO, Bill and Melinda Gates Foundation, Pediatric health services, Kampala, Uganda, Brick-and-mortar facility, Satellite clinics, Investment Portfolio
\end{IEEEkeywords}

\section{Introduction}
The provision of pediatric health services in Kampala, Uganda, is a critical issue that requires immediate attention. The city’s growing population and the surrounding areas have increased the demand for these services, highlighting the need for a sustainable solution. This report outlines a strategic plan for an NGO that has received a grant from the Bill and Melinda Gates Foundation to address this issue over a ten-year period.
The project involves the construction of a central healthcare facility in Kampala and 15 satellite clinics within a 100-mile radius. Each clinic will be staffed with two doctors and three Physician Assistants, paid the median local annual wage which is . 23555.41 sud for doctors and 15228.39 usd for physician assistants.The operation will have an initial total capital expense of $5MM, with supply and utility costs estimated at USD 10k/month.
The financial strategy for this project is threefold. First, we identify the capital and operating costs required for the venture over the ten-year term. Second, we structure an endowment, excluding the capital required, that will generate enough annual revenues to meet the necessary operating expenses and a 20% yearly reinvestment payment. The initial endowment face value must be repaid to the Gates Foundation at the end of the term, with any remaining balance used to ensure ongoing operations. Finally, we develop an investment portfolio based on historical risky asset pricing data from 2018 - 2022 and risk-free auction data from 2022. The performance of this portfolio will be evaluated based on 2023 data, with reallocation done if the portfolio is not projected to meet the NGO’s obligations.
In conclusion, this project presents a viable and sustainable solution to the pressing need for pediatric health services in Kampala, Uganda. By leveraging the grant from the Bill and Melinda Gates Foundation and implementing a robust financial strategy, the NGO can ensure its continuous operation and contribution to the community’s health and well-being beyond the ten-year grant period. Future perspectives include potential expansion of the project to other regions in Uganda, further enhancing the country’s healthcare infrastructure.

\section{Results}
Describes the results of the study here.
The projected financial requirements for the pediatric healthcare initiative in Kampala, Uganda, reveal a one-year operating expense of $212,796. Over the ten-year term, the cumulative operating expenses amount to $3,404,736. When considering both capital and operating costs for the entire period, the total investment needed is estimated at $18,404,736. Employing a continuous discounting approach with a 5% discount rate, the present value of the total costs for the ten-year term is calculated to be $11,298,911. This present value provides a comprehensive assessment, accounting for the time value of money, and serves as a key metric in determining the financial feasibility and sustainability of the healthcare project.
The projected financial framework underscores an annual revenue requirement of $1,840,473 from the endowment to sustain the pediatric healthcare initiative in Kampala, Uganda. Accordingly, the initial grant size for the first year aligns precisely with this annual revenue requirement, totaling $1,840,473. As the organization strategically plans for financial sustainability beyond the initial grant, the available funds for investment, calculated at $13,251,410, form a crucial foundation for constructing a robust investment portfolio.

Informed by historical performance, the chosen portfolio includes the following tickers: "PFE", "MRK", "AMD", "MU", "INTC", "AAPL", "GOOGL", "MSFT", "TSLA", "AMZN", and "SPY". Through meticulous simulation and reallocation, it was discerned that the adjusted portfolio significantly outperformed the benchmark (SPY) by a substantial margin. This outperformance positions the portfolio as a formidable means to meet all necessary financial obligations for the healthcare initiative.

To fortify the endowment's safety further, the NGO will commence the allocation of funds to Treasury Inflation-Protected Securities (TIPS) from the second year onwards. This strategic decision is grounded in the unique attributes of TIPS, providing protection against inflation by adjusting the principal value based on changes in the Consumer Price Index for All Urban Consumers (CPI-U). In times of inflation, the principal value of TIPS increases, offering a hedge against rising costs, whereas in deflationary scenarios, the principal value decreases, ensuring a level of financial security for the endowment.

% Figures: Change me as needed. Figures need to be PDF format
\begin{figure}[!h]\centering
\includegraphics[width=0.43\textwidth]{/Users/moitra/Desktop/Screenshot 2023-12-17 at 3.28.51 AM.png}
\end{figure}


\section{Discussion}
The discussion should be three paragraphs (or less). 
\begin{itemize}
\item{
  1.	The financial analysis and endowment structure for the pediatric healthcare initiative in Kampala, Uganda, yield promising outcomes with significant implications for the initiative's sustainability. The study successfully determines an annual revenue requirement of $1,840,473, aligning the initial grant size precisely with the financial demands of the project. This targeted funding approach ensures that the endowment is well-positioned to cover ongoing operational costs and strategic reinvestment, laying a solid foundation for the initiative's long-term success. The investment portfolio, strategically constructed with equities based on historical performance, not only outperforms the benchmark but also presents a reliable source for meeting future financial obligations. Furthermore, the inclusion of Treasury Inflation-Protected Securities (TIPS) in the risk mitigation strategy adds an additional layer of financial security, safeguarding against inflation and enhancing the overall resilience of the endowment. In essence, the results underscore a comprehensive and adaptive financial plan, aligning with the goals of the Bill and Melinda Gates Foundation grant and ensuring the sustained impact of the healthcare initiative in Kampala.
  }
\item{2.	Analysis in the Context of Alternative Approaches:
In evaluating the financial framework and endowment structure for the pediatric healthcare initiative in Kampala, Uganda, it is imperative to contextualize our study in relation to alternative approaches. While our chosen strategy relies on a judicious combination of annual revenue requirements, investment portfolios, and risk mitigation through TIPS, alternative models could include variations in investment instruments, diverse asset allocations, or alternative financial instruments. Some approaches may prioritize a more conservative investment portfolio, emphasizing stable but lower returns, while others might opt for a more aggressive strategy with a higher risk tolerance to potentially achieve greater yields. Moreover, alternative approaches might explore different financial instruments beyond TIPS for inflation protection. By acknowledging the existence of various financial paradigms, our study recognizes the inherent complexities of financial planning and seeks to emphasize the adaptability of our chosen strategy in achieving the long-term sustainability of the healthcare initiative.

}
\item{Despite the robustness of our financial analysis, there are inherent limitations that warrant consideration. First and foremost, the predictive nature of financial markets introduces an element of uncertainty. While historical performance data informs our investment portfolio, future market behavior remains unpredictable. Additionally, the continuous discounting approach employed assumes a stable discount rate over the ten-year term, yet real-world scenarios may witness fluctuations in interest rates. Furthermore, the reliance on historical data may not fully capture unprecedented events or economic shifts, introducing an element of risk in portfolio performance. Another limitation lies in the assumption of continuous funding from the Gates Foundation; any deviation from this assumption could impact the study's outcomes. Lastly, the effectiveness of risk mitigation with TIPS is contingent upon the accuracy of inflation projections, and unexpected variations could influence the efficacy of this strategy. Recognizing these limitations provides a nuanced perspective on the study's findings and underscores the importance of ongoing monitoring and adaptability in financial planning}
\end{itemize}

\section{Materials and Methods}
1.	First we loaded the historical dataset SP500-Daily-OHLC-1-3-2018-to-11-17-2023.jld2. Now we Clean the data. Not all of the tickers in our dataset have the maximum number of trading days for various reasons, e.g., acquistion or de-listing events. Let's collect only those tickers with the maximum number of trading days. First, let's compute the number of records for a company that we know has a maximim value, e.g., AAPL and save that value.
2.	We also import these additional files Additional Files:
•	CapitalAllocationLine-NewCo-PD1-CHEME-5660-Fall-2023.jld2
•	CapitalAllocationLine-NewCo-PD1-CHEME-5660-Fall-2023.pdf
•	CapitalAllocationLine-Solo Jacket-PD1-CHEME-5660-Fall-2023.jld2
•	CapitalAllocationLine-Solo Jacket-PD1-CHEME-5660-Fall-2023.pdf
•	EfficientFrontier-NewCo-PD1-CHEME-5660-Fall-2023.jld2
•	EfficientFrontier-NewCo-PD1-CHEME-5660-Fall-2023.pdf
•	EfficientFrontier-PortfolioDriftExample-PD1-CHEME-5660-Fall-2023.jld2
•	EfficientFrontier-Solo Jacket-PD1-CHEME-5660-Fall-2023.jld2
•	EfficientFrontier-Solo Jacket-PD1-CHEME-5660-Fall-2023.pdf
•	SIMs-Solo Jacket-PD1-CHEME-5660-Fall-2023.jld2
•	SIMs-PortfolioDriftExample-PD1-CHEME-5660-Fall-2023.jld2
•	SIMs-NewCo-PD1-CHEME-5660-Fall-2023.jld2
3.	The projected financial requirements for the pediatric healthcare initiative in Kampala, Uganda, were meticulously calculated and assessed to ensure a robust financial foundation. The one-year operating expense is estimated at $212,796, with a cumulative operating cost of $3,404,736 over the ten-year term. Considering both capital and operating costs for the entire period, the total investment needed is projected at $18,404,736. Employing a continuous discounting approach with a 5% discount rate, the present value of the total costs over the ten-year term was calculated as $11,298,911. This present value, accounting for the time value of money, serves as a pivotal metric in determining the financial feasibility and sustainability of the healthcare project
4.	The financial framework highlights an annual revenue requirement of $1,840,473 from the endowment to sustain the pediatric healthcare initiative. Consequently, the initial grant size for the first year precisely aligns with this annual revenue requirement, totaling $1,840,473. As part of a strategic plan for financial sustainability beyond the initial grant, the available funds for investment, calculated at $13,251,410, form a crucial foundation for constructing a robust investment portfolio.
5.	Informed by historical performance, a carefully curated portfolio was constructed, featuring the following tickers: "PFE," "MRK," "AMD," "MU," "INTC," "AAPL," "GOOGL," "MSFT," "TSLA," "AMZN," and "SPY." Through meticulous simulation and strategic reallocation, it was discerned that the adjusted portfolio significantly outperformed the benchmark (SPY) by a substantial margin. This outperformance positions the portfolio as a formidable means to meet all necessary financial obligations for the healthcare initiative.
6.7.	To fortify the endowment's safety further, the NGO will implement a strategic allocation of funds to Treasury Inflation-Protected Securities (TIPS) starting from the second year onwards. This decision is grounded in the unique attributes of TIPS, offering protection against inflation by adjusting the principal value based on changes in the Consumer Price Index for All Urban Consumers (CPI-U). In times of inflation, the principal value of TIPS increases, providing a hedge against rising costs. Conversely, in deflationary scenarios, the principal value decreases, ensuring a level of financial security for the endowment. This dual mechanism enhances the endowment's resilience against economic fluctuations and reinforces its ability to fulfill its mission over the long term.

\section{Data and model availability}
The computational framework for this study utilized Julia (version 1.9.3)  for implementing the model equations, , The code is available at https://github.com/bm687/CHEME-5660-Project-Template-F23.


% References -
\bibliographystyle{naturemag_noURL}
\bibliography{Project}
Bezanson, J., Edelman, A., Karpinski, S. & Shah, V. B. Julia: A fresh
approach to numerical computing (2014). arXiv: 1411.1607

\end{document}
\grid